% Options for packages loaded elsewhere
\PassOptionsToPackage{unicode}{hyperref}
\PassOptionsToPackage{hyphens}{url}
%
\documentclass[
]{book}
\usepackage{lmodern}
\usepackage{amssymb,amsmath}
\usepackage{ifxetex,ifluatex}
\ifnum 0\ifxetex 1\fi\ifluatex 1\fi=0 % if pdftex
  \usepackage[T1]{fontenc}
  \usepackage[utf8]{inputenc}
  \usepackage{textcomp} % provide euro and other symbols
\else % if luatex or xetex
  \usepackage{unicode-math}
  \defaultfontfeatures{Scale=MatchLowercase}
  \defaultfontfeatures[\rmfamily]{Ligatures=TeX,Scale=1}
\fi
% Use upquote if available, for straight quotes in verbatim environments
\IfFileExists{upquote.sty}{\usepackage{upquote}}{}
\IfFileExists{microtype.sty}{% use microtype if available
  \usepackage[]{microtype}
  \UseMicrotypeSet[protrusion]{basicmath} % disable protrusion for tt fonts
}{}
\makeatletter
\@ifundefined{KOMAClassName}{% if non-KOMA class
  \IfFileExists{parskip.sty}{%
    \usepackage{parskip}
  }{% else
    \setlength{\parindent}{0pt}
    \setlength{\parskip}{6pt plus 2pt minus 1pt}}
}{% if KOMA class
  \KOMAoptions{parskip=half}}
\makeatother
\usepackage{xcolor}
\IfFileExists{xurl.sty}{\usepackage{xurl}}{} % add URL line breaks if available
\IfFileExists{bookmark.sty}{\usepackage{bookmark}}{\usepackage{hyperref}}
\hypersetup{
  pdftitle={Tutorial For CPT Tool},
  pdfauthor={Jiangdong Liu},
  hidelinks,
  pdfcreator={LaTeX via pandoc}}
\urlstyle{same} % disable monospaced font for URLs
\usepackage{longtable,booktabs}
% Correct order of tables after \paragraph or \subparagraph
\usepackage{etoolbox}
\makeatletter
\patchcmd\longtable{\par}{\if@noskipsec\mbox{}\fi\par}{}{}
\makeatother
% Allow footnotes in longtable head/foot
\IfFileExists{footnotehyper.sty}{\usepackage{footnotehyper}}{\usepackage{footnote}}
\makesavenoteenv{longtable}
\usepackage{graphicx,grffile}
\makeatletter
\def\maxwidth{\ifdim\Gin@nat@width>\linewidth\linewidth\else\Gin@nat@width\fi}
\def\maxheight{\ifdim\Gin@nat@height>\textheight\textheight\else\Gin@nat@height\fi}
\makeatother
% Scale images if necessary, so that they will not overflow the page
% margins by default, and it is still possible to overwrite the defaults
% using explicit options in \includegraphics[width, height, ...]{}
\setkeys{Gin}{width=\maxwidth,height=\maxheight,keepaspectratio}
% Set default figure placement to htbp
\makeatletter
\def\fps@figure{htbp}
\makeatother
\setlength{\emergencystretch}{3em} % prevent overfull lines
\providecommand{\tightlist}{%
  \setlength{\itemsep}{0pt}\setlength{\parskip}{0pt}}
\setcounter{secnumdepth}{5}
\usepackage{booktabs}
\usepackage{amsthm}
\makeatletter
\def\thm@space@setup{%
  \thm@preskip=8pt plus 2pt minus 4pt
  \thm@postskip=\thm@preskip
}
\makeatother
\usepackage[]{natbib}
\bibliographystyle{apalike}

\title{Tutorial For CPT Tool}
\author{Jiangdong Liu}
\date{2020-03-20}

\begin{document}
\maketitle

{
\setcounter{tocdepth}{1}
\tableofcontents
}
\hypertarget{intended-use}{%
\chapter{Intended Use}\label{intended-use}}

The Gulf Conservation Prioritization Tool (CPT) is \textbf{not} intended to be prescriptive. Instead this tool was designed to provide data to \textbf{support} conservation planning efforts across the Gulf Coast Region. All users acknowledge that the CPT model is intended to \textbf{explore} ecological and socioeconomic co-benefits of proposed areas of land conservation, and should \textbf{not} be used in a decision making context.

The flexibility of this tool enables a user to evaluate conservation alternatives using either a \textbf{multi-criteria decision analysis (MCDA)} framework, or user-defined values.

This book is intended to provide additional support documents for the data measures used in the CPT tool.

\hypertarget{intro}{%
\chapter{Introduction}\label{intro}}

Again, some placeholders for some handle thing we could add-in later.

You can label chapter and section titles using \texttt{\{\#label\}} after them, e.g., we can reference Chapter \ref{intro}. If you do not manually label them, there will be automatic labels anyway, e.g., Chapter \ref{habitat}.

You can write citations, too. For example, we are using the \textbf{bookdown} package \citep{chuck2019} in this sample book, which was built on top of R Markdown and \textbf{knitr} \citep{xie2015}.

\hypertarget{cpt-opening-page}{%
\chapter{CPT Opening Page:}\label{cpt-opening-page}}

\hypertarget{video}{%
\section{Video}\label{video}}

\hypertarget{written-documents}{%
\section{Written documents}\label{written-documents}}

There are three options to choose from.\\
Single-Project Mode, Multi-Project Mode, and Portfolio Mode.
Select Single-Project mode if you want to do location characterization
Multi-project mode if you want to compare areas of interest for conservation value.
Portfolio mode if you are in the future and have the ability to use this option.

\hypertarget{selection-of-project-footprint-page}{%
\chapter{Selection of Project Footprint Page:}\label{selection-of-project-footprint-page}}

\hypertarget{video-1}{%
\section{Video}\label{video-1}}

\hypertarget{written-documents-1}{%
\section{Written documents}\label{written-documents-1}}

On this page, you will need to define your Area(s) of Interest. All areas of interest have to be defined within the SCA region, which is outlined in blue. There are three ways to define the Area of Interest (AOI).\\
The first way is to draw a polygon. To draw a polygon, start by clicking once on the pentagon icon located on the left of the screen. Now drawing mode is activated. Then move the mouse to where you want your AOI, and click and drag to the mouse to create the border of the AOI. To finish an AOI, click on the first point you made for your AOI, which deactivates drawing mode. Then click on the button `Finalize area of interest' located in the top-right panel of the screen. If you are dissatisfied, or made a mistake with drawing your AOI, you have the option to either edit the drawn shape, using the `edit layer' icon, or delete it entirely using the `trashcan' icon, both located on the left of the screen. The third way is to select a watershed boundary as your AOI. All watersheds are HUC12. In the dropdown menu for `Input' located in the top-right panel of the screen, choose `Select from a HUC-12 watershed'. You will then have the option to select a watershed by its name, its 12-digit ID, or its physical boundary via the `Watershed Selection Options' dropdown menu located in the top-right panel of the screen. If you choose to select a watershed by its name or 12-digit ID, you will use the `Select a watershed' dropdown menu located at the top-right panel of the screen to select. If you choose to select a watershed's physical boundary, the map will render all HUC-12 watershed boundaries (takes a few seconds), and you can use your cursor to click on the watershed of your choice. The selected watershed boundary will change from blue to red. After you have correctly selected your desired watershed, click `Finish' located in the top-right panel of the screen.\\
The third way is to upload a shapefile of your AOI. To upload a shapefile, click on the drop-down menu for `Input' located in the top-right panel of the screen, and select `Input from .zip boundary shapefile'. You will then be prompted to browse your computer for your desired .zip folder. The browse icon will appear in the top-right panel of the screen. Once you have selected and uploaded a shapefile, click `Finish' located in the top-right panel.

\hypertarget{single-project-mode}{%
\chapter{Single-Project Mode:}\label{single-project-mode}}

\hypertarget{video-2}{%
\section{Video}\label{video-2}}

\hypertarget{written-documents-2}{%
\section{Written documents}\label{written-documents-2}}

Once you have finalized an Area of Interest, you will be taken to the `Assessment' page, where you can view the characteristics of your AOI.\\
If you'd like to create a downloadable summary report of your AOI, you may do so by clicking on the `Detailed Report' button located in the top-right corner of this page.\\
You also have the ability to download a shapefile of your AOI by clicking on the `Spatial footprint' button, also located in the top-right corner of this page.
You can also download all spatial data for your AOI by clicking on the `Export Raw Datatable' button located in the top-right corner of this page.
If you would like to do more your AOI, click on the `Advanced Options' button on the right side of the screen, and three additional options will appear: `Rename AOI', `Refine AOI', and `Turn on all supporting layers'.
By clicking `Rename AOI', you will be taken to a new tab where you can type your desired AOI name in the text input box.\\
By clicking `Refine AOI', another tab will appear with a map of your AOI will be visible. To refine the AOI, hover your mouse over the `layers' icon in the top right of the map, and check the box next to `View Hexagons'. Once the hexagons are visible, you can remove hexagons from your defined AOI by first clicking on the hexagons you want removed. The selected hexagons for removal will turn from blue to grey. To officially remove the selected hexagons, click on the `Deselect Highlighted Hexagons' button, located on the right side of the screen.\\
By clicking `Turn on all supporting layers', the map of the AOI in the bottom right of the screen will make the following layers visible: NERR Conservation Areas, FNAI BOT Conservation Areas, MS Conservation Opportunity Areas, LA Conservation Opportunity Areas, AL Conservation Opportunity Areas, and the SECAS Blueprint. To turn on or off any of these layers, hover your mouse over the `layers' icon in the top-right of the map, and check/uncheck any layers you desire to hide or view.

\hypertarget{multi-project-mode}{%
\chapter{Multi-Project Mode:}\label{multi-project-mode}}

\hypertarget{video-3}{%
\section{Video}\label{video-3}}

\hypertarget{written-documents-3}{%
\section{Written documents}\label{written-documents-3}}

Once you have selected your AOIs, you will be taken a new page called `Data Overview' where you will see the raw values for each ecological/socioeconomic measure, organized by RESTORE Goal. From here you will have three options to move forward: `Default Results', `Choose my own weights', or `simulate random weights'.
If you select `Show Default Results' you will be taken to a new window that shows the overall scores and goal-specific scores of each AOI, assuming all RESTORE goals are equally weighted (i.e.~all at 20\%). To return to the main menu, click `Return to Main Menu' button located on the right side of the screen, and to select different weights for your AOI comparison, click the `If you would like to alter weights in your comparison' button, also located in the right side of the screen.\\
If you select `Choose my own weights', you will be taken to a new screen to adjust the weights for the RESTORE Goals, and then for each measure within the RESTORE Goals. You can adjust each goal weight by adjusting the slider bar or numeric input box next to each goal. A summary of all goal weights is visible in a table to the right of the screen, with the total sum at the bottom of the table. The sum of all weights must equal 100\%. Once all goal weights are at their desired values, click the `Finalize Weights for Goals' button located underneath the goal weights summary table. You will then be taken to a new window to select weights for each measure, organized by each RESTORE Goal under separate tabs. Simply select low, medium, high, or zero as weights for each measure (each measure is pre-selected to medium), and after you have gone through each measure, you will be taken to a final tab labeled `Weights review', where you can view a table of all measures and their selected weights for your comparison. Click the `Finalize Weights' button and you will be taken to a results page that shows the overall scores and goal-specific scores of each AOI given the user-selected weights.\\
On the results page you will then have the option to download a summarized report of your AOI comparisons by clicking the `Generate Detailed Report' button, located underneath the summary table. You can also click the `Download Spatial Footprint' button, also located underneath the summary table, to download the shapefiles of each AOI.
If you select `Simulate Random Weights', a pop-up window will appear that says `Please give up to 5 minutes for the model to run. This model will run 10,000 iterations through MCDA model. Meanwhile, please do not make changes to the application.' Click `OK' in the pop-up window to continue with the random weight simulation. When the simulation has started, a new pop-up window will appear alerting you of it beginning, and a progress bar will appear to show you how many iterations are left before the simulation is complete. When the simulation is complete, you will automatically be taken to the `Results' tab, where you will see the Rank acceptability graphs for each AOI, and the Central Weights Vector graph. To see the rank acceptability graph for a different AOI, select from the dropdown menu above the pie charts labeled `Select a proposal'. To learn more about how to interpret Central Weight Vectors, click the `How can I use Central Weights' button to download a pdf document that contains more information. You also have the option to download a summarized report of your AOI comparisons by clicking the `Generate Detailed Report' button, located in the center of the page. You can also click the `Download Spatial Footprint' button, also located in the center of the page, to download the shapefiles of each AOI.

\hypertarget{advanced-options-for-multi-project-mode}{%
\chapter{Advanced options for Multi-project mode:}\label{advanced-options-for-multi-project-mode}}

\hypertarget{video-4}{%
\section{Video}\label{video-4}}

\hypertarget{written-documents-4}{%
\section{Written documents}\label{written-documents-4}}

There are seven advanced features available in multi-project mode: `View datatable', `Refine Area of Interest', `Customize Utility Function', `Add Attribute', `Rename Area of Interest', `Adjust the number of iterations run by the MCDA model', and `Turn on all the supporting layers'.
Selecting `View datatable' will allow the user to see the raw data for each AOI.\\
By clicking `Refine AOI', another tab will appear with a map of your AOI will be visible. To refine the AOI, hover your mouse over the `layers' icon in the top right of the map, and check the box next to `View Hexagons'. Once the hexagons are visible, you can remove hexagons from your defined AOI by first clicking on the hexagons you want removed. The selected hexagons for removal will turn from blue to grey. To officially remove the selected hexagons, click on the `Deselect Highlighted Hexagons' button, located on the right side of the screen.\\
By clicking `Customize Utility Function', the user will be able to adjust the way a particular measure translates the raw data into a utility value used in scoring a project.\\
By clicking `Rename AOI', you will be taken to a new tab where you can type your desired AOI name in the text input box.\\
By clicking `Turn on all supporting layers', the map of the AOI in the bottom right of the screen will make the following layers visible: NERR Conservation Areas, FNAI BOT Conservation Areas, MS Conservation Opportunity Areas, LA Conservation Opportunity Areas, AL Conservation Opportunity Areas, and the SECAS Blueprint. To turn on or off any of these layers, hover your mouse over the `layers' icon in the top-right of the map, and check/uncheck any layers you desire to hide or view.

  \bibliography{book.bib,packages.bib}

\end{document}
